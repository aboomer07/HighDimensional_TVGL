\documentclass{beamer}
\usetheme{Boadilla}
\usepackage{tikz}
 \usepackage[english]{babel}
 \usepackage{amsmath}
 \usepackage{amssymb}
 \usepackage{amsthm}
 \usepackage{mathtools} 
 \usepackage[utf8]{inputenc}
 \usepackage{dsfont}
 \usetikzlibrary{patterns}
 \usepackage{mathrsfs}
 \usepackage{bbold}
\usetikzlibrary{mindmap,trees,shadows}
\usepackage{caption}
\usepackage{tabularx}
\usepackage{booktabs}
\usepackage{apacite}
\usepackage{listings}
\captionsetup[figure]{font=scriptsize}
\newcommand{\MYhref}[3][blue]{\href{#2}{\color{#1}{#3}}}%
\usepackage{hyperref} 
            \hypersetup{backref=true,       
                    pagebackref=true,               
                    hyperindex=true,                
                    colorlinks=true,                
                    breaklinks=true,                
                    urlcolor= blue,                
                    linkcolor= blue,                
                    bookmarks=true,                 
                    bookmarksopen=false,
                    citecolor=blue,
                    linkcolor=blue,
                    filecolor=blue,
                    citecolor=blue,
                    linkbordercolor=blue
}

\definecolor{background}{RGB}{39, 40, 34}
\definecolor{string}{RGB}{230, 219, 116}
\definecolor{comment}{RGB}{117, 113, 94}
\definecolor{normal}{RGB}{248, 248, 242}
\definecolor{identifier}{RGB}{166, 226, 46}

\lstset{
  language = SQL,  % choose the language of the code
  linewidth = 12cm,
  numbers = left, % where to put the line-numbers
  stepnumber=1,  % the step between two line-numbers.
  numbersep=5pt, % how far the line-numbers are from the code
  numberstyle=\tiny\color{black}\ttfamily,
  backgroundcolor=\color{background}, % choose the background color. You must add \usepackage{color}
  showspaces=false, % show spaces adding particular underscores
  showstringspaces=false,             % underline spaces within strings
  showtabs=false, % show tabs within strings adding particular underscores
  tabsize=4,                          % sets default tabsize to 2 spaces
  captionpos=b,                       % sets the caption-position to bottom
  breaklines=true,                    % sets automatic line breaking
  breakatwhitespace=true, % sets if automatic breaks should only happen at whitespace
  title=\lstname,  % show the filename of files included with \lstinputlisting;
  basicstyle=\color{normal}\ttfamily,     % sets font style for the code
  keywordstyle=\color{magenta}\ttfamily,  % sets color for keywords
  stringstyle=\color{string}\ttfamily,    % sets color for strings
  commentstyle=\color{comment}\ttfamily,  % sets color for comments
  emph={format_string, eff_ana_bf, permute, eff_ana_btr},
  emphstyle=\color{identifier}\ttfamily,
  belowskip=-0.5cm
}

\title[High Dimensional Models] % (optional, only for long titles)
{High Dimensional Models}
\subtitle{Time-Varying Graphical Lasso}
\author[Andrew Boomer \& Jacob Pichelmann] % (optional, for multiple authors)
{Andrew Boomer \& Jacob Pichelmann}
\institute []
{Toulouse School of Economics \\ M2 EEE}

\date{\today}

\graphicspath{
    {.} % document root dir
    {./Output/}
    {./Report/}
}

\begin{document}


\frame{\titlepage}
    
\begin{frame}{Introduction to Graphs}
    \begin{itemize}
        \item A graph is represented by a set of vertices $V = \{1, 2, \dotsb, p\}$ and by a set of edges $E \subset V x V$
        \item A graph is undirected when there is no distinction between the edge $(s, t)$ and $(t, s) \quad \forall \ s, t \in E$
        \item Graphs can also be thought of in terms of Markov chains, and some of the properties and representations of Markov Chains apply to graphs as well
        \item For example, the edge set of a graph can be represented as a Markov transition matrix. For an edge set $E \subset (a, b) x (a, b)$ the matrix would be:
        \[\begin{bmatrix} W_{(a, a)} & W_{(a, b)} \\
           W_{(b, a)} & W_{(b, b)} \end{bmatrix}\]
    \end{itemize}
\end{frame}

\begin{frame}{Extension to Graphical Models}
  \begin{itemize}
    \item This graphical representation can be extended to a high dimensional set of random variables $X_{s}$.
    \item In this example, $s$ corresponds to a single vertex with the whole vertex set $V$ of the total graph. The connections between each vertex in the Markov transition matrix representation quanity the relationship between this random variables.
  \end{itemize}
  \begin{figure}
  \begin{minipage}{0.4\textwidth}
      The two in Figure \ref{fig:IntroPic} are equivalent, where white spaces indicate no relationship, and grey spaces indicate a 1-to-1 relationship.
  \end{minipage}%
  \begin{minipage}{0.6\textwidth}
    \begin{figure}
       \includegraphics[width=6cm]{IntroGraphMat.png}
       \caption{Graph and Sparsity Matrix}
       \label{fig:IntroPic}
  \end{figure} 
  \end{minipage}
  \end{figure}
\end{frame}

\begin{frame}{Gaussian Graphical Models}
  \begin{itemize}
    \item For a gaussian graphical model, we start from a gaussian distribution with dimension equal to the number of vertices $X \sim \mathcal{N}(\mu, \Sigma)$
    \item The sparsity matrix in the slide above, for a gaussian graphical model, is given by $\Theta = \Sigma^{-1}$, and is also known as the precision matrix. The goal of a gaussian graphical model is to solve for $\Theta$.
    \item The gaussian distribution can be rewritten in terms of a maximum likelihood problem to solve for $\hat{\Theta}_{MLE}$, where $S$ is the empirical covariance matrix, rearranged and simplified to:
    \[\mathcal{L}(\Theta; X) = log \ det \ \Theta - trace(S \Theta)\]
    \item This MLE problem converges to the true precision matrix when $N \xrightarrow{} \infty$. The MLE problem is, however, not even feasible if $p > N$, where $p$ is the number of dimensions.
  \end{itemize}
\end{frame}

\begin{frame}{Graphical LASSO}
  \begin{itemize}
    \item As with a LASSO in the context of a linear regression, adding a regularization term can solve the issue of a non-full rank matrix. Additionally, the LASSO term induces low-importance terms to zero.
    \item In the case of the graphical LASSO, weak edge connections in the precision matrix will go to zero, increasing the sparsity of the resulting solution. This increases sparsity aids in interpretability by filtering out noisy relationships, and can be used even when $N$ is close to $p$.
  \end{itemize}
\end{frame}

\begin{frame}{Challenge: The Network Structure Can Change Over Time}
    In many real world settings (e.g. financial markets) the structure of the complex system changes over time.
    Intersecting time series analysis and network science allows detecting anomalies, spotting trends and forecasting future behavior.
    \begin{figure}
       \includegraphics[width=6cm]{NetworkGraph.png}
       \caption{}
       % \source{\cite(hallac2017network)}
       \label{fig:network_evolution}
  \end{figure}
\end{frame}

\begin{frame}{Conclusion \& Limitations}
    The assumption of $X \sim \mathcal{N}(\mu, \Sigma)$ is unlikely to hold in some settings.
\end{frame}

\begin{frame}{References}
    \nocite{*}
    \bibliographystyle{apacite}
    \bibliography{References}
\end{frame}







\end{document}